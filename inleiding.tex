%%=============================================================================
%% Inleiding
%%=============================================================================

\chapter{Inleiding}
\label{ch:inleiding}
Deze bachelorproef handelt over Citrix XenApp en Citrix XenDesktop. Citrix XenApp en Citrix XenDesktop bieden een veilige omgeving aan voor virtuele applicaties en desktops. Ik ga dieper in op de verschillen tussen deze twee producten en wil hierbij de case van Orbid oplossen. Orbid is een bedrijf dat 90 medewerkers telt en ik wil nagaan welk product voor hun het interessantst is.

Dankzij applicatievirtualisatie in de vorm van Citrix XenApp kunnen Windowsapplicaties benaderd worden vanaf elk toestel met een internetverbinding. De applicaties worden dus gehost op een gedeelde computer waarbij de gebruiker enkel moet beschikken over Citrix Receiver en een internetvebrinding. Citrix Receiver is de client van Citrix die u toegang verleent tot XenApp- en Xendesktop-omgevingen.

XenDesktop is een voorbeeld van VDI, oftewel Virtual Desktop Infrastructure. Je krijgt dus toegang tot een volledige desktop die dus ergens gehost wordt. XenApp en XenDesktop kunnen perfect gecombineerd worden. Het plaatsen van krachtige systemen die dan benaderd worden op afstand heeft als voordeel dat je makkelijk software kan updaten en dat je dit overal ter wereld kan uitvoeren. Je hoeft niet naar de eindgebruiker zelf te gaan om het probleem op te lossen of de update uit te voeren, je kan het probleem overal ter wereld oplossen doordat de applicaties en de desktops nu eenmaal centraal beheerd worden.

In deze bachelorproef wordt er gebruikt gemaakt van een Citrix Workspace die mij is aangeboden door Mathias Alleyn(Senior Engineer bij Citrix). In deze workspace kan je XenApp, XenDesktop en nog vele andere Citrix producten testen in een gehoste omgeving.

De onderzoeksvragen waarrond deze bachelorproef draait zijn dus:
\begin{itemize}
\item Wat zijn de verschillen tussen Citrix XenApp en Citrix XenDesktop?
\item Welk Citrix product is voor Orbid het interessantst?
\end{itemize}
	
Het spreekt voor zich dat er eerst op zoek wordt gegaan naar de mogelijkheden en de verschillen tussen deze 2 producten alvorens er kan gekeken worden naar welk product er nu het meest geschikt is voor Orbid.




\section{Stand van zaken}
\label{sec:stand-van-zaken}

%% TODO: deze sectie (die je kan opsplitsen in verschillende secties) bevat je
%% literatuurstudie. Vergeet niet telkens je bronnen te vermelden!

Citrix Systems
Citrix is een softwarebedrijf dat server, applicatie, desktopvirtualisatie en software as a service(SaaS) levert. Citrix werd opgericht in 1989 door wijlen Edward E. lacobucci in Richardson, Texas.

Geschiedenis Citrix




Wat is VDI?
Virtual Desktop Infrastructure oftewel VDI is een techniek waarbij desktops in een virtuele omgeving gedraaid worden. Deze desktops draaien op een server. Je kan de desktop bereiken via een thin-client, pc, laptop of tablet. Een belangrijke opmerking hierbij is dat het lijkt alsof je bezig bent met een lokale installatie. Een voorbeeld van Virtual Desktop Infrastructure is in dit geval XenDesktop, hierbij gaat het om desktopvirtualisatie. XenApp is per definitie geen voorbeeld van VDI, het is namelijk een voorbeeld van applicatievirtualisatie, die applicaties draaien natuurlijk ook op een desktop waardoor je het ook kan zien als een soort van desktopvirtualisatie.

Citrix Workspace
Alle tests worden uitgevoerd in een Citrix Workspace. Mijn testomgeving omvat:
\begin{itemize}
	\item XenApp/XenDesktop 7.15
	\item XenMobile 10.7
	\item NetScaler 12.0 (Build 53.13)
	\item StoreFront 3.12
\end{itemize}
Hosted Desktops en VDI's
\begin{itemize}
	\item Windows Server 2012
	\item Windows Serevr 2016
	\item Windows 7
	\item Windows 10
\end{itemize}
Hosted Applications:
\begin{itemize}	
	\item Office 2013/2016
	\item Healthcare
	\item Financial
	\item Sales
	\item Admin Tools
\end{itemize}
Mobile Applications:
\begin{itemize}
	\item Secure Mail
	\item Secure Web
	\item Secure Notes
	\item Secure Tasks
	\item ShareFile
	\item Receiver
	\item ShareFile StorageZones Controller 5.1
	\item AppDNA 7.15
\end{itemize}
Platform details :
\begin{itemize}
	\item Citrix XenServer 7.0
	\item 128 GB Ram
	
\end{itemize}

AFBEELDING TONEN



\section{Probleemstelling en Onderzoeksvragen}
\label{sec:onderzoeksvragen}

%% TODO:
%% Uit je probleemstelling moet duidelijk zijn dat je onderzoek een meerwaarde
%% heeft voor een concrete doelgroep (bv. een bedrijf).
%%
%% Wees zo concreet mogelijk bij het formuleren van je
%% onderzoeksvra(a)g(en). Een onderzoeksvraag is trouwens iets waar nog
%% niemand op dit moment een antwoord heeft (voor zover je kan nagaan).
Dit onderzoek moet het voor bedrijven makkelijker maken om te kiezen tussen Citrix XenApp of Citrix XenDesktop. Het maken van een juiste keuze is belangrijk omdat XenDesktop veel duurder is dan XenApp. Een specifieke case heb ik gevonden in het bedrijf Orbid dat 90 medewerkers telt. De onderzoeksvragen zijn:

\begin{itemize}
\item Wat zijn de verschillen tussen Citrix XenApp en Citrix XenDesktop?
\item Welk Citrix product is voor Orbid het interessantst?
\end{itemize}

\section{Opzet van deze bachelorproef}
\label{sec:opzet-bachelorproef}

%% TODO: Het is gebruikelijk aan het einde van de inleiding een overzicht te
%% geven van de opbouw van de rest van de tekst. Deze sectie bevat al een aanzet
%% die je kan aanvullen/aanpassen in functie van je eigen tekst.

De rest van deze bachelorproef is als volgt opgebouwd:

In Hoofdstuk~\ref{ch:methodologie} wordt de methodologie toegelicht en worden de gebruikte onderzoekstechnieken besproken om een antwoord te kunnen formuleren op de onderzoeksvragen.

%% TODO: Vul hier aan voor je eigen hoofstukken, één of twee zinnen per hoofdstuk

In Hoofdstuk~\ref{ch:conclusie}, tenslotte, wordt de conclusie gegeven en een antwoord geformuleerd op de onderzoeksvragen. Daarbij wordt ook een aanzet gegeven voor toekomstig onderzoek binnen dit domein.

