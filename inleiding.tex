%%=============================================================================
%% Inleiding
%%=============================================================================

\chapter{Inleiding}
\label{ch:inleiding}
Deze bachelorproef handelt over Citrix XenApp en Citrix XenDesktop. Citrix XenApp en Citrix XenDesktop bieden een veilige omgeving aan voor virtuele applicaties en desktops. Ik ga dieper in op de verschillen tussen deze twee producten en wil hierbij de case van Orbid oplossen. Orbid is een bedrijf dat 90 medewerkers telt en ik wil nagaan welk product voor hun het interessantst is.

Dankzij applicatievirtualisatie in de vorm van Citrix XenApp kunnen Windowsapplicaties benaderd worden vanaf elk toestel met een internetverbinding. De applicaties worden dus gehost op een gedeelde computer waarbij de gebruiker enkel moet beschikken over Citrix Receiver en een internetvebrinding. Citrix Receiver is de client van Citrix die u toegang verleent tot XenApp- en Xendesktop-omgevingen.

XenDesktop is een voorbeeld van VDI, oftewel Virtual Desktop Infrastructure. Je krijgt dus toegang tot een volledige desktop die dus ergens gehost wordt. XenApp en XenDesktop kunnen perfect gecombineerd worden. Het plaatsen van krachtige systemen die dan benaderd worden op afstand heeft als voordeel dat je makkelijk software kan updaten en dat je dit overal ter wereld kan uitvoeren. Je hoeft niet naar de eindgebruiker zelf te gaan om het probleem op te lossen of de update uit te voeren, je kan het probleem overal ter wereld oplossen doordat de applicaties en de desktops nu eenmaal centraal beheerd worden.

In deze bachelorproef wordt er gebruikt gemaakt van een Citrix Workspace die mij is aangeboden door Mathias Alleyn(Senior Engineer bij Citrix). In deze workspace kan je XenApp, XenDesktop en nog vele andere Citrix producten testen in een gehoste omgeving.

De onderzoeksvragen waarrond deze bachelorproef draait zijn dus:
\begin{itemize}
\item Wat zijn de verschillen tussen Citrix XenApp en Citrix XenDesktop?
\item Welk Citrix product is voor Orbid het interessantst?
\end{itemize}
	
Het spreekt voor zich dat er eerst op zoek wordt gegaan naar de mogelijkheden en de verschillen tussen deze 2 producten alvorens er kan gekeken worden naar welk product er nu het meest geschikt is voor Orbid.


\section{Stand van zaken}
\label{sec:stand-van-zaken}

%% TODO: deze sectie (die je kan opsplitsen in verschillende secties) bevat je
%% literatuurstudie. Vergeet niet telkens je bronnen te vermelden!

Citrix XenApp en Citrix XenDesktop zijn 2 producten van Citrix. Vooraleer ik dieper inga op deze 2 producten, wil ik wat achtergrondinformatie meegeven over dit bedrijf.
\subsection{Citrix Systems}
Citrix Systems, Inc. is een internationaal Amerikaans IT-bedrijf en is in deze industrie 1 van de grootste namen. Citrix werd opgericht op 17 april 1989 door wijlen Edward E. Lacobucci in Richardson, Texas. Edward werkte hiervoor als softwareontwikkelaar bij IBM. Het hoofdkantoor van Citrix bevindt zich in Fort Lauderdale in Texas. Citrix telt ongeveer 8100 werknemers en had in 2016 een omzet van maar liefst 3,42 miljard dollar.

Citrix was niet altijd even succesvol. In de beginjaren was distributed computing nu eenmaal de norm. Ze werden zelfs beschuldigd van het heruitvinden van het mainframe en van teruggaan in de tijd. Hierdoor overleefde Citrix amper de beginjaren. Het is correct om te stellen dat Citrix z’n tijd ver vooruit was door zich al voor 2000 bezig te houden met server-based computing en gecentraliseerde applicaties en data.
\begin{figure}
	\caption{David Henshall, CEO van Citrix}
	\footcite{Citrix2017e}
	\includegraphics{davidhenshall}
\end{figure}
De CEO van Citrix Systems is sinds juli 2017 David J. Henshall. David had hiervoor al ervaring als COO en CFO van Citrix. In 2013 en 2014 heeft hij even als tijdelijke CEO gewerkt voor Citrix. David bekleedt tevens een positie als lid van de raad van bestuur in LogMeIn en Everbridge. Het is natuurlijk niet toevallig dat die bedrijven zijn die enerzijds RDS en in security voorzien. In het verleden werkte hij bij Cypress Semiconductor en Samsung.


\section{Probleemstelling en Onderzoeksvragen}
\label{sec:onderzoeksvragen}

%% TODO:
%% Uit je probleemstelling moet duidelijk zijn dat je onderzoek een meerwaarde
%% heeft voor een concrete doelgroep (bv. een bedrijf).
%%
%% Wees zo concreet mogelijk bij het formuleren van je
%% onderzoeksvra(a)g(en). Een onderzoeksvraag is trouwens iets waar nog
%% niemand op dit moment een antwoord heeft (voor zover je kan nagaan).
Dit onderzoek moet het voor bedrijven makkelijker maken om te kiezen tussen Citrix XenApp of Citrix XenDesktop. Het maken van een juiste keuze is belangrijk omdat XenDesktop veel duurder is dan XenApp. Een specifieke case heb ik gevonden in het bedrijf Orbid dat 90 medewerkers telt. De onderzoeksvragen zijn:

\begin{itemize}
\item Wat zijn de verschillen tussen Citrix XenApp en Citrix XenDesktop?
\item Welk Citrix product is voor Orbid het interessantst?
\end{itemize}

\section{Opzet van deze bachelorproef}
\label{sec:opzet-bachelorproef}

%% TODO: Het is gebruikelijk aan het einde van de inleiding een overzicht te
%% geven van de opbouw van de rest van de tekst. Deze sectie bevat al een aanzet
%% die je kan aanvullen/aanpassen in functie van je eigen tekst.

De rest van deze bachelorproef is als volgt opgebouwd:

In Hoofdstuk~\ref{ch:methodologie} wordt de methodologie toegelicht en worden de gebruikte onderzoekstechnieken besproken om een antwoord te kunnen formuleren op de onderzoeksvragen.

%% TODO: Vul hier aan voor je eigen hoofstukken, één of twee zinnen per hoofdstuk

In Hoofdstuk~\ref{ch:conclusie}, tenslotte, wordt de conclusie gegeven en een antwoord geformuleerd op de onderzoeksvragen. Daarbij wordt ook een aanzet gegeven voor toekomstig onderzoek binnen dit domein.

